% This is lnbip.tex the demonstration file of the LaTeX macro package for
% Lecture Notes in Business Information Processing from Springer-Verlag.
% It serves as a template for authors as well.
% version 1.0 for LaTeX2e
%
\documentclass[lnbip]{svmultln}
%
\usepackage{makeidx}  % allows for indexgeneration
\usepackage{amsfonts,amssymb}
\usepackage{mathrsfs}
\usepackage{enumitem}
\usepackage{amsmath}
\usepackage{circuitikz}
\usepackage{tikz}


% \makeindex          % be prepared for an author index
%
\begin{document}
%

\mainmatter              % start of the contribution
%
\title{Simplified Square Span Program}
%
\author{}%Ivar Ekeland\inst{1} \and Roger Temam\inst{2}
%
\institute{}%Princeton University, Princeton NJ 08544, USA,\\

\maketitle              % typeset the title of the contribution
%
\section{Background}
\subsection{Span Programs}
A Span Program (SP) is a linear-algebraic model of computation introduced by Karchmer and Wigderson\cite{gennaro2013quadratic}\cite{karchmer1993span}.
\begin{definition}
 A SP over a field $\mathbb{F}$ consists of a nonzero target vector $\textbf{t}$ over $\mathbb{F}$, a set of vectors $\mathcal{V}=\{\textbf{v}_1,...,\textbf{v}_m\}$, a partition of the indices $\mathcal{I}=\{1,...,m\}$ into two sets $\mathcal{I}_{labeled}$ and $\mathcal{I}_{free}$, and a further partition of $\mathcal{I}_{labeled}$ as $\cup_{i \in [n],j \in \{0,1\}}\mathcal{I}_{ij}$. 
\end{definition}
The SP is said to ``compute'' a function $f$ if the following is true for all input assignment $u \in \{0,1\}^n$: the target vector is in the span of the vectors that ``belong'' to the input assignments $u$ -- namely, the vectors with indices in $\mathcal{I}_u = \mathcal{I}_{free}\cup_i \mathcal{I}_{i,u_i}$ -- iff $f(u)=1$. The size of the span program is $m$.

% \subsection{Boolean Function}

\subsection{Circuit Checker Function}
Suppose $C$ is a Boolean circuit that computes a function $f$.
\begin{definition}
Let $f:\{0,1\}^n \rightarrow \{0,1\}$ be a function whose Boolean circuit $C$ has $s$ gates. Let $N=n+s$. Suppose $\phi:\{0,1\}^N \rightarrow \{0,1\}$ is a function that outputs `1' iff the input is a valid assignment of $C$'s wires (wires that fan out are considered one wire) with output wire set to `1'. We say that $\phi$ is the circuit checker function for $f$.
\end{definition}

% \subsection{Square Span Programs (SSPs)}
% \begin{definition}
% A \textit{square span program} $Q$ over the field $\mathbb{F}$ consists of $m+1$ polynomials $v_0(x),v_1(x),...,v_m(x)$ and a target polynomial $t(x)$ such that $deg(v_i(x)) \leq deg(t(x))$ for all $i=0,...,m$.

% We say that the square span program $Q$ has size $m$ and degree $d=deg(t(x))$.

% We say that $Q$ accepts an input $(a_1,...,a_l)\in \mathbb{F}_l$ if and only if there exist $a_{l+1},...,a_m \in \mathbb{F}$ satisfying
% $$t(x)\quad divides\quad \left(v_0(x)+\sum_{i=1}^ma_iv_i(x)\right)^2-1.$$

% We say that $Q$ verifies a boolean function f:$\{0,1\}^l \rightarrow \{0,1\}$ if it accepts exactly those inputs $\textbf{a} \in \mathbb{F}^l$ that satisfy $\textbf{a} \in \{0,1\}^l $ and $f(\textbf{a})=1$.
% \end{definition} 

\section{Verifiable Computation}

A public verifiable computation (VC) scheme allows a computationally limited client to outsource the computation of a function $F$ on input $u$ to an untrusted worker, and then verify the correctness of the returned result $F(u)$. Critically, the outsourcing and verification procedures must be significantly more efficient for the client than performing the computation by itself.\cite{parno2012delegate} \

A public verifiable computation scheme provides \textit{public delegation,} which allows arbitrary parties to submit inputs for delegation; and \textit{public verifiability,} which allows arbitrary parties (not just the delegator) to verify the correctness of the results returned by the worker.
The following definition captures these two properties.
\begin{definition} 
A public verifiable computation scheme $\mathcal{VC}$ consists of a set of three polynomial-tima algorithms {\rm(KeyGen, Compute, Verify)} defined as follows:
\begin{itemize}
\item[$\bullet$] $(EK_F,VK_F) \leftarrow {\rm KeyGen}(F, 1^\lambda)$: The randomized key generation algorithm takes the function $F$ to be outsourced and security parameter $\lambda$. It outputs a public evaluation key $EK_F$, and a public verification key $VK_F$

\item[$\bullet$] $(y,\pi_y) \leftarrow {\rm Compute}(EK_F,u)$: The deterministic worker algorithm uses the public evaluation key $EK_F$ and input $u$. It outputs $y \leftarrow F(u)$ and a proof $\pi_y$ of y's correctness.

\item[$\bullet$] $\{0,1\} \leftarrow {\rm Verify}(VK_F,u,y,\pi_y)$: Given the verification key $VK_F$, the deterministic verification algorithm outputs 1 if $F(u) = y$, and 0 otherwise.
\end{itemize}

\end{definition}


% Prior work gives formal definitions for correctness, security, and efficiency:
% \begin{itemize}
%   \item[$\bullet$]\textbf{Correctness}
%   \item[$\bullet$]\textbf{Security}
%   \item[$\bullet$]\textbf{Efficiency}
% \end{itemize}

\section{Simplified Square Span Program}
We define SSSP somewhat similarly to SSP.
\begin{definition}
A simplified square span program (SSSP) $Q$ over the field $\mathbb{F}$ consists of $m+1$ polynomials $v_0(x),v_1(x),...,v_m(x)$ and a target polynomial $t(x)$ such that $deg(v_i(x)) \leq deg(t(x)) $ for all $i=0,...,m$.

$$t(x) \quad divides\quad \left(\sum\limits_{i=0}^ma_i v_i(x)\right)^2-1$$

Specifically, $a_0=1$. 

We say that Q verifies a boolean function $f:\{0,1\}^l \rightarrow \{0,1\}$ if it accepts exactly those imputs $\textbf{a} \in \mathbb{F}^l$ that satisfy $\textbf{a} \in \{0,1\}^l$ and $f(\textbf{a})=1$. We may see $f$ as a binary circuit.

\end{definition}

\subsection{Transformation From Circuit Satisfiability to SSSPs}
Consider a circuit consisting of two NAND gates, as is shown below:\\
\begin{center}
\begin{circuitikz} \draw
(0,1) node[nand port] (myor1) {}
    (myor1.in 1) node [anchor=east] {$a_1$}
    (myor1.in 2) node [anchor=east] {$a_2$}
    (myor1.out) node [anchor=west] {$a_3$}
(2,0) node[nand port] (myand1) {}
    % (myand1.in 2) node [anchor=east] {C}
    (myand1.out) node [anchor=west] {$a_5$}
    (myor1.out) to (myand1.in 1);

\draw (myand1.in 2) to (myand1.in 2 -| myor1.in 2)
        node [anchor=east] {$a_4$};
\end{circuitikz}
\end{center}

It satisfies that
$$a_3 = \neg (a_1\wedge a_2)$$
$$a_5 = \neg (a_3\wedge a_4)$$

To guarantee $a_1,a_2,...,a_5 \in \{0,1\}$, we use the constraints
$$(2a_i-1)^2=1,\quad i=1,2,...,5$$

To linearize\cite{groth2012new} the NAND gate with input $a_1,a_2$ and output $a_3$, writing $\bar c$ for $1-c$, we have


\begin{align*}
a_3 = \neg (a_1\wedge a_2) & \iff a_1 +a_2-2\bar a_3 \in  \{0,1\} \\
& \iff \left(2(a_1 +a_2-2(1-a_3))-1\right)^2=1 \\
& \iff \left(2a_1 +2a_2+4a_3-5)\right)^2=1 
\end{align*} 

Similarly, we have
$$ \left(2a_3 +2a_4+4a_5-5)\right)^2=1$$

% Here, $\textbf{a}=(a_1,a_2,a_3,a_4,a_5)$
The satisfiability of the circuit can therefore be represented by 7 quadratic equations:

$$(2a_1-1)^2 =1 \quad (2a_2-1)^2 =1 \quad ... \quad (2a_5-1)^2 =1$$
$$(2a_1+2a_2+4a_3-5)^2=1$$
$$(2a_3+2a_4+4a_5-5)^2=1$$
Corresponding to $(\textbf{a} V)^2=1$.\\

Then we can represent the constraints as 
$$\textbf{a}V=(1,a_1,a_2,a_3,a_4,a_5)
\begin{pmatrix} 
-1 & -1 & -1 & -1 & -1 & -5 & -5 \\
2 & 0 & 0 & 0 & 0 & 2 & 0 \\
0 & 2 & 0 & 0 & 0 & 2 & 0 \\
0 & 0 & 2 & 0 & 0 & 4 & 2 \\
0 & 0 & 0 & 2 & 0 & 0 & 2 \\
0 & 0 & 0 & 0 & 2 & 0 & 4 
\end{pmatrix}
$$


To get a SSSP, let $p$ be a prime and $r_1,r_2,...,r_7$ be 7 distinct elements in $\mathbb{Z}_p$. Pick degree 5 polynomials $v_0(x),v_1(x),...,v_5(x)$ such that
$$
\begin{pmatrix} 
% v_0(r_1) & v_0(r_2) & v_0(r_3) & v_0(r_4) & v_0(r_5) & v_0(r_6) & v_0(r_7) \\
% v_1(r_1) & v_1(r_2) & v_1(r_3) & v_1(r_4) & v_1(r_5) & v_1(r_6) & v_1(r_7) \\
% v_2(r_1) & v_2(r_2) & v_2(r_3) & v_2(r_4) & v_2(r_5) & v_2(r_6) & v_2(r_7) \\
% v_3(r_1) & v_3(r_2) & v_3(r_3) & v_3(r_4) & v_3(r_5) & v_3(r_6) & v_3(r_7) \\
% v_4(r_1) & v_4(r_2) & v_4(r_3) & v_4(r_4) & v_4(r_5) & v_4(r_6) & v_4(r_7) \\
% v_5(r_1) & v_5(r_2) & v_5(r_3) & v_5(r_4) & v_5(r_5) & v_5(r_6) & v_5(r_7) \\

v_0(r_1) & v_0(r_2) & \dots & v_0(r_7) \\
v_1(r_1) & v_1(r_2) & \dots & v_1(r_7) \\
\vdots   & \ddots   &       & \vdots   \\
v_5(r_1) &          & \ddots& v_5(r_7) 
\end{pmatrix}
= V =
\begin{pmatrix} 
-1 & -1 & -1 & -1 & -1 & -5 & -5 \\
2 & 0 & 0 & 0 & 0 & 2 & 0 \\
0 & 2 & 0 & 0 & 0 & 2 & 0 \\
0 & 0 & 2 & 0 & 0 & 4 & 2 \\
0 & 0 & 0 & 2 & 0 & 0 & 2 \\
0 & 0 & 0 & 0 & 2 & 0 & 4 
\end{pmatrix}
$$

Let $t(x)=(x-r_1)(x-r_2)...(x-r_7)$ to get a SSSP $(v_0(x),v_1(x),...,v_5(x),t(x))$ for the circuit such that
$$t(x)\quad divides\quad \left(\sum\limits_{i=0}^5a_i v_i(x)\right)^2-1, \quad where ~ a_0=1$$
If and only if $a_1,a_2,...a_5$ satisfy the circuit.


\section{Building Verifiable Computation from SSSPs}
% \subsection{Non-interactive zero-knowledge arguments of knowledge}


\subsection{Bilinear groups}
We use the following notation\cite{boneh2005evaluating}:
\begin{itemize}
	\item[1.] $\mathbb{G}$ and $\mathbb{G}_T$ are two (multiplicative) cyclic groups of prime order $q$.
	\item[2.] G is a generator of $\mathbb{G}$.
	\item[3.] $e$ is a bilinear map $e:\mathbb{G} \times \mathbb{G} \rightarrow \mathbb{G}_T$. That is, for all $U,V \in \mathbb{G}$ and $a,b \in \mathbb{Z}$, we have $e(U^a,V^b)=e(U,V)^{ab}.$ We also require that $e(G,G)$ is a generator of $\mathbb{G}_T$.
\end{itemize}

We say that $\mathbb{G}$ is a \textbf{bilinear group} if there exists a group $\mathbb{G}_T$ and a bilinear
map as above.

\subsection{Verifiable Computation from SSSPs}
We will now construct succinct and perfect NIZK argument of knowledge for any functions $l_u, l_w$ and families $\{\mathcal{R}\}_\lambda$ of relations $R$ of pairs $(u,w) \in \{0,1\}^{l_u(\lambda)} \times \{0,1\}^{l_w(\lambda)} $ that can be computed by polinomial size circuits with $m(\lambda)$ wires and $n(\lambda)$ gates for a total size of $d(\lambda)=m(\lambda)+n(\lambda)$ \\

\begin{itemize}
\item[$\bullet$] $(\sigma,\tau) \leftarrow$ Setup$(1^\lambda,R)$: Run $gk := (p,\mathbb{G},\mathbb{G}_T,e) \leftarrow \mathcal{G}(1_\lambda)$. Parse $R$ as a boolean circuit $C_R:\{0,1\}^{l_u}\times \{0,1\}^{l_w} \rightarrow \{0,1\}$. Generate a SSSP $Q = (v_0(x),...,v_m(x),t(x))$ that verifies $C_R$ over $\mathbb{Z}_p$. Pick $G \leftarrow \mathbb{G}^*$ and $\beta,s \leftarrow \mathbb{Z}_p^*$ such that $t(s) \neq 0$. Return
\begin{eqnarray*}
  \sigma &=& (gk,G,...G^{s^d},\{G^{\beta v_i(s)}\}_{i>l_u},G^{\beta t(s)},Q)\\
  \tau &=& (\sigma,\beta,s)
\end{eqnarray*}

\item[$\bullet$] $\pi \leftarrow$ Prove$(\sigma, u,w)$: Parse $u$ as $(a_1,...,a_{l_u}) \in \{0,1\}^{l_u}$ and use $w$ to compute $a_{l_u+1},...,a_m$ such that $t(x)$ divides $\left(\sum\limits_{i=0}^ma_i v_i(x)\right)^2-1$.

Let 
$$h(x) = \frac{\left(\sum\limits_{i=0}^ma_i v_i(x)\right)^2-1}{t(x)}$$

Use linear combinations of the elements in $\sigma$ to compute
\begin{alignat*}{2}
 H &= G^{h(s)}  &\quad V_w &= G^{\sum\limits_{i>l_u}^m a_iv_v(s)} \\  
 B_w &= G^{\beta\big(\sum\limits_{i>l_u}^m a_iv_i(s)\big)} & 
 V &= G^{\sum\limits_{i=0}^m a_iv_i(s)}
\end{alignat*}

and return $\pi=(H,V_w,B_w,V)$.\\

\item[$\bullet$] $\{0,1\} \leftarrow$ Verify$(\sigma,u,\pi)$: Parse $u$ as $(a_1,...,a_{l_u})\in \{0,1\}^{l_u}$ and $\pi$ as $(H,V_w,B_w,V) \in \mathbb{G}^4$. Return 1 if and only if
%e(V,\hat{G})=e(G,\hat{V})\quad 
$$e(H,G ^{t(s)})=e(V^2/G) \quad e(V_w,G^\beta)=e(B_w,G).  $$
\end{itemize}

\section{Extractor}
......

% \begin{itemize}
% \item[$\bullet$] $(\sigma,\tau) \leftarrow$ Setup$(1^\lambda,R)$: Run $gk := (p,\mathbb{G}, \hat{\mathbb{G}},\mathbb{G}_T,e) \leftarrow \mathcal{G}(1_\lambda)$. Parse $R$ as a boolean circuit $C_R:\{0,1\}^{l_u}\times \{0,1\}^{l_w} \rightarrow \{0,1\}$. Generate a SSSP $Q = (v_0(x),...,v_m(x),t(x))$ that verifies $C_R$ over $\mathbb{Z}_p$. Pick $G \leftarrow \mathbb{G}^*$ and $\hat{G} \leftarrow \hat{\mathbb{G}}^*$ and $\beta,s \leftarrow \mathbb{Z}_p^*$ such that $t(s) \neq 0$. Return
% \begin{eqnarray*}
%   \sigma &=& (gk,G,\hat{G},...G^{s^d},\hat{G}^{s^d},\{G^{\beta t(s)}\}_{i>l_u},G^{\beta t(s)},Q)\\
%   \tau &=& (\sigma,\beta,s)
% \end{eqnarray*}

% \item[$\bullet$] $\pi \leftarrow$ Prove$(\sigma, u,w)$: Parse $u$ as $(a_1,...,a_{l_u}) \in \{0,1\}^{l_u}$ and use $w$ to compute $a_{l_u+1},...,a_m$ such that $t(x)$ divides $\left(\sum\limits_{i=0}^ma_i v_i(x)\right)^2-1$.

% Pick $\delta \leftarrow \mathbb{Z}_p$ and let 
% $$h(x) = \frac{\left(\sum\limits_{i=0}^ma_i v_i(x)+\delta t(x)\right)^2-1}{t(x)}$$

% Use linear combinations of the elements in $\sigma$ to compute
% \begin{alignat*}{2}
%  H &= G^{h(s)}  &\quad V_w &= G^{\sum\limits_{i>l_u}^m a_iv_v(s)+\delta t(s)} \\  
%  %B_w &= G^{\beta(\sum\limits_{i>l_u}^m a_iv_v(s)+\delta t(s))} & 
%  \hat{V}
%     &= \hat{G}^{\sum\limits_{i=0}^m a_iv_i(s)+\delta t(s)}
% \end{alignat*}

% and return $\pi=(H,V_w,\hat{V})$.

% \item[$\bullet$] $\{0,1\} \leftarrow$ Verify$(\sigma,u,\pi)$: Parse $u$ as $(a_1,...,a_{l_u})\in \{0,1\}^{l_u}$ and $\pi$ as $(H,V_w,\hat{V}) \in \mathbb{G}^3\times \hat{\mathbb{G}}$. Compute $V=G^{\sum\limits_{i=0}^{l_u}a_iv_i(s)}V_w$ and return 1 if and only if
% $$e(V,\hat{G})=e(G,\hat{V})\quad e(H,\hat{G}^{t(s)})=e(V,\hat{V})e(G,\hat{G})^{-1}  $$
% \end{itemize}

% $\pi \leftarrow$ Sim$(\tau,u)$: Parse $u$ as $(a_1,...,a_{l_u})\in \{0,1\}^{l_u}$ and pick $\delta_w \leftarrow \mathbb{Z}_p$ at random.
% Let
% $$h=\frac{\left(\sum\limits_{i=0}^{l_u} a_iv_v(s)+\delta_w\right)^2-1}{t(s)}$$
% and return $\pi = (G_h,G_{\delta_w},G_{\beta\delta_w},\hat{G}^{\sum\limits_{i=0}^{l_u} a_iv_v(s)+\delta_w})$.

% \section{Conclusion}
%
\bibliographystyle{plain}
\bibliography{sssp}

\end{document}
